\documentclass[a4paper,10.5pt]{jsarticle}
\usepackage{url}
\usepackage{color}
\usepackage[dvipdfmx]{graphicx}
\setcounter{secnumdepth}{3}
\makeatletter
\def\@maketitle{
\begin{flushright}
{\large \@date}
\end{flushright}
\begin{center}
{\LARGE \@title \par}
\end{center}
\begin{flushright}
{\large \@author}
\end{flushright}
\par\vskip 1.5em
}
\makeatother
\usepackage{ascmac}
\usepackage{color}
\usepackage{amssymb}
\usepackage[margin=20truemm]{geometry}
\usepackage{mathtools}
\usepackage[version=3]{mhchem}
\usepackage{latexsym} 
\usepackage{otf}
\pagestyle{myheadings}


\begin{document}

\title{脳神経外科 追再試験レポート}
\author{02-221071 鄭 禹亨}
\date{}
\maketitle

\section{急性硬膜外血腫について以下の項目について概説せよ}
\begin{enumerate}
  \item \emph{原因}\par 急性硬膜外血腫ができる原因は主に頭部外傷による頭蓋骨骨折が挙げられる。
  頭部外傷になる原因としては、高所・階段からの落下や交通事故、スポーツ中の頭部への衝撃などが挙げられる。
  \item \emph{出血源となる血管}\par 硬膜と頭蓋骨の間にある血管。具体的には、中硬膜動脈などの硬膜動脈や静脈洞などが挙げられる。
  \item \emph{画像所見}\par CTでは、硬膜と頭蓋骨の間にある血腫を確認することができる。具体的には、凸レンズ状の高吸収領域(白色)が確認できる。
  よく鑑別に上がる似た疾患として、硬膜下血腫が挙げられるが、硬膜下血腫はCTでは三日月型の高吸収領域(白色)が確認できる。
  \item \emph{臨床症状}\par 硬膜外血腫の特徴的な臨床症状として、意識清明期がある。意識清明期とは、受傷後一時的に意識が回復する期間のことで、見かけ上治ったように見える。
  だが硬膜外血腫の場合、その後意識が急速に低下することが多いので、病院に運ばれた際に意識が清明だったとしても退院させずにCT検査をすることが重要である。
  初期症状として、急激な頭痛、嘔吐、意識障害などがあり、救急外来で運ばれてくることが多い。その後、意識清明期を経て、治療が遅れた場合は再度意識障害を起こし、致命率が急激に高まる。
  \item \emph{\textup{治療法(内科的治療、外科的治療)}}\par 内科的な保存的治療として、患者の安静を保ち、頭部の適切な位置を維持し、鎮痛剤や抗てんかん薬を使用することが多い。
  血液凝固の異常がある場合はその治療も行う。
  だが、治療の主流は外科的に血腫を摘出することであり、開頭血腫除去術とよばれる。開頭血腫除去術によって脳の圧迫を軽減し、神経損傷を防ぐ必要がある。
  また、手術後にモニタリングを行い、合併症の早期発見をする必要がある。注意すべき合併症として、脳ヘルニアによる脳幹圧迫症状がある。脳幹が圧迫されると生命維持中枢が侵される場合もあるので早めの血腫除去が大事になってくる。
  また、除去が遅れて脳が圧迫され続けると血腫を除去した後も高次障害が残ることがあるので、そこにも留意する。
\end{enumerate}


\end{document}